% Overview:
%   Conclusioni TeX subfile for the project.
%   Each subfile MUST start with the following line
%		\documentclass[../main.tex]{subfiles}

\documentclass[../main.tex]{subfiles}

\begin{document}

\section{Conclusioni}

L'analisi dei dati prodotti dalle tecnologie NGS consente di migliorare la nostra comprensione della diversità genetica e delle relative malattie genetiche. In questa tesina abbiamo presentato e analizzato alcuni tool \textit{alignment-free} dell'attuale stato dell'arte che effettuano la genotipizzazione e rilevano mutazioni all'interno del genoma: questi sono stati presentati in opposizione ai tool che effettuano l'allineamento delle read del donatore in input e la sequenza del genoma di riferimento che, tuttavia, posseggono alcuni punti negativi, precedentemente discussi, prima tra tutti l'elevata complessità temporale. I framework \textit{alignment free} sono stati dettagliatamente analizzati per comprendere i motivi della loro efficienza, funzionalità e performance nel Capitolo \ref{frameworks}.

Abbiamo in precedenza già sottolineato la diversità degli elementi all'interno di questa categoria di framework: i metodi \textit{alignment-free} sono accomunati dal non ricorrere alla tecnica dell'allineamento delle read ma utilizzano diverse strutture dati e distinti algoritmi efficienti; per citare qualche esempio, alcuni si basano su conteggi di \textit{k}-mer (parole), altri sulla costruzione di grafi e altri ancora sulla teoria dell'informazione. Abbiamo cercato, nella nostra selezione iniziale dei tool, di scegliere un rappresentativo insieme di metodi che contribuisco all'attuale stato dell'arte. I primi quattro metodi considerati utilizzano infatti anche un genoma di riferimento in input (\textit{reference-based}), diversamente dai successivi due. In particolare Kevlar \cite{standage2019kevlar} oltre al sample di read, richiede in input read provenienti dai genomi dei due genitori, per convalidare la presenza di mutazioni \textit{de novo}, presenti nel figlio ma assenti nei genitori, mentre DiscoSnp\texttt{++} \cite{peterlongo2017discosnp++} è un metodo ibrido che effettua l'individuazione delle varianti senza l'utilizzo della reference in input, la quale può essere facoltativamente utilizzata nell'ultimo step per individuare e mappare le mutazioni predette sul genoma di riferimento. Inoltre la caratteristica che contraddistingue DiscoSnp\texttt{++} dagli altri è che non è basato su parole ma sui grafi, in particolare i grafi di \textit{de Bruijn} e rileva le varianti analizzando la struttura del grafo e la presenza di bolle. Un'altra differenza riportata all'interno dei framework selezionati riguarda il tipo di varianti genomiche su cui i metodi si concentrano: COBASI \cite{gomez-romero2018cobasi} e Kevlar si specializzano nel rilevare mutazioni \textit{de novo}, MALVA  \cite{bernardini2019malva} si concentra su indel, anche di grandi dimensioni, e SNP multi-allelici mentre gli altri solo su SNP e piccoli indel.

In generale i framework \textit{alignment-free} si propongono come una soluzione efficiente al problema della genotipizzazione, sono più veloci dei framework che effettuano l'allineamento, poiché l'allineamento in generale è un'operazione difficile, e quindi sono computazionalmente meno costosi, pur mantenendo un'accuratezza comparabile; inoltre, alcuni problemi dei metodi \textit{alignment-based}, come la scalabilità o l'incapacità di gestire riarrangiamenti vengono in questo modo risolte. I tool sono accomunati inoltre dall'utilizzare, nella fase finale di chiamata delle varianti, un modello probabilistico per determinare la probabilità di tutti i possibili genotipi e assegnare poi, a ciascuna variante, quello con la maggiore probabilità.

Una considerazione che possiamo fare è che allo stato attuale non c'è un tool \textit{alignment-free} che prevale sugli altri o che si occupa del problema nella sua interezza: chiaramente ognuno dei metodi analizzati risolve in maniera efficiente e con elevata precisione un determinato task ma ognuno possiede alcuni aspetti che possono essere migliorati, per coprire più task o comunque un task più generale e/o essere più performante. Alcune soluzioni potrebbero essere ottimizzare ulteriormente le strutture dati per richiedere minor spazio o variare il modello probabilistico utilizzato durante la fase finale di genotipizzazione.

Concludiamo quindi affermando che, sebbene i tool attualmente proposti possano essere migliorati, i framework \textit{alignment-free} possono essere considerati efficienti ed efficaci nel risolvere il problema della genotipizzazione e della rilevazione di mutazioni del genoma. 
































\end{document}