% Overview:
%   Conclusioni TeX subfile for the project.
%   Each subfile MUST start with the following line
%		\documentclass[../main.tex]{subfiles}

\documentclass[../main.tex]{subfiles}

\begin{document}

\section{Conclusioni}

\textbf{Scrivere conclusioni}

L'analisi dei dati prodotti dalle tecnologie NGS consente di migliorare la nostra comprensione della diversità genetica e delle relative malattie genetiche. In questa tesina abbiamo presentato e analizzato alcuni tool \textit{alignment-free} per effettuare la genotipizzazione e rilevare mutazioni all'interno del genoma: sono stati dettagliatamente analizzati per comprendere la loro efficienza, funzionalità e performance. 

Abbiamo in precedenza già sottolineato la diversità degli elementi all'interno di questa categoria di framework: i metodi \textit{alignment-free} sono accomunati dal non ricorrere alla tecnica dell'allineamento delle read, ma utilizzano diverse strutture dati e distinti algoritmi; per citare qualche esempio, alcuni si basano su conteggi di \textit{k}-mer (parole), altri sulla costruzione di grafi e altri sulla teoria dell'informazione. Abbiamo cercato, nella selezione iniziale dei tool, di scegliere un rappresentativo insieme di tool che contribuisco all'attuale stato dell'arte. I primi quattro metodi considerati utilizzano un genoma di riferimento in input (\textit{reference-based}) mentre i successivi due no. In particolare Kevlar \cite{standage2019kevlar} oltre al sample di read, richiede in input, read provenienti dai genomi dei due genitori, per convalidare la presenza di mutazioni \textit{de novo}, presenti nel figlio ma assenti nei genitori. Mentre DiscoSnp\texttt{++} \cite{peterlongo2017discosnp++} effettua l'individuazione delle varianti completamente senza l'utilizzo della reference, che può essere facoltativamente utilizzata nell'ultimo step per individuare e mappare le mutazioni predette sul genoma di riferimento. Inoltre la caratteristica che contraddistingue DiscoSnp\texttt{++} dagli altri è che non è basato su parole ma sui grafi, in particolare i grafi di \textit{de Bruijn} e rileva le varianti analizzando la struttura del grafo e la presenza di bolle. Un'altra differenza riportata all'interno dei framework selezionati riguarda il tipo di varianti genomiche su cui i metodi si concentrano: COBASI \cite{gomez-romero2018cobasi} e  Kevlar si specializzano nel rilevare mutazioni \textit{de novo}, MALVA  \cite{bernardini2019malva} si concentra su indel, anche di grandi dimensioni, e SNP multi-allelici e gli altri solo su SNP e piccoli indel.

In generale i framework \textit{alignment-free} sono più veloci dei framework che effettuano l'allineamento, poiché l'allineamento in generale è un'operazione difficile, e quindi sono computazionalmente meno costosi; inoltre, alcuni problemi dei metodi \textit{alignment-based}, come la scalabilità o l'incapacità di gestire riarrangiamenti vengono in questo modo risolte. Anche se tutti i tool attualmente proposti possono essere migliorati e implementati per gestire un numero maggiore di casi, possiamo concludere validando l'efficienza e l'efficacia dei tool \textit{alignment-based} nel risolvere il problema della genotipizzazione e della rilevazione di mutazioni del genoma. 

\textcolor{red}{OVVIAMENTE DA MODIFICARE}






























\end{document}