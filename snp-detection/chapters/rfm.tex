% Overview:
%   Introduzione TeX subfile for the project.
%   Each subfile MUST start with the following line
%		\documentclass[../main.tex]{subfiles}

\documentclass[../main.tex]{subfiles}

\begin{document}

\section{Reference Free Frameworks}
\label{rfm}

\textcolor{red}{
Il grafico de Bruijn di una serie di sequenze di DNA o RNA
è una struttura di dati che svolge un ruolo sempre più importante nelle applicazioni di sequenziamento di prossima generazione. Tuttavia, un importante problema pratico di questa struttura è la sua elevata impronta di memoria per i grandi organismi. Ad esempio, il codifica semplice del grafico de Bruijn per l'essere umano il genoma (n $circa$ 2,4 · 109, dimensione k-mer k = 27) richiede 15 GB (n · k / 4 byte) di memoria per memorizzare le sequenze di nodi solo. Grafici per genomi e metagenomi molto più grandi non può essere costruito su un tipo. TODO: Migliorare e ampliare introduzione sui refence free frameworks.}

I metodi reference-based sopracitati, non possono essere applicati quando non è presente un genoma di riferimento. 
In generale, i metodi reference-free specializzati nell’individuare SNP possono essere divisi in due macro categorie: 
\begin{itemize} 
\item i metodi della prima categoria eseguono inizialmente un assemblaggio de novo (de novo assembly) per costruire una sequenza di riferimento. Successivamente tramite un sotto-modulo (subroutine) reference based per mappare le letture (reads) di ogni individuo su questa nuova sequenza di riferimento. Ci si riferisce a questi metodi come ibridi (hybrid) siccome utilizzano sia assemblaggio de novo sia tecniche di mappatura (mapping techniques) per individuare SNP.
\item i metodi della seconda categoria si concentrano direttamente sull’assemblaggio degli SNP, senza cercare di assemblare preventivamente un riferimento completo (full reference)
\end{itemize}

\end{document}