% Overview:
%   Introduzione TeX subfile for the project.
%   Each subfile MUST start with the following line
%		\documentclass[../main.tex]{subfiles}

\documentclass[../main.tex]{subfiles}

\begin{document}

\section{Reference Based Frameworks}
\label{rbm}

Rilevare SNP (Single Nucleotide Polymorphisms) e mutazioni tra genomi sta diventando una pratica molto comune nell'ambito della NGS (Next-generation sequencing). In generale, i primi metodi di rilevamento SNP utilizzavano un genoma di riferimento: come precedentemente citato, i metodi basati sull'allineamento includono la mappatura delle read in input con il genoma di riferimento, una procedura costosa. 

Pertanto negli ultimi anni sono stati proposti strumenti senza allineamento: la maggior parte degli attuali framework \textit{alignment free} utilizza ugualmente un genoma di riferimento e una lista preassegnata di SNP noti ed utilizza questi due input per chiamare le varianti all'interno delle read del donatore. Ci sono anche altri framework e modelli che riescono a identificare SNP e determinare il genotipo, senza l'utilizzo di una \textit{reference} e di cui parleremo in seguito (vedi Sezione \ref{rfm}). \\

In questa prima sezione si riporteranno e analizzeranno alcuni dei più attuali e significativi metodi che non utilizzano la costosa pratica di allineamento delle read ma che usano un genoma di riferimento per determinare mutazioni ed SNP ed effettuare la genotipizzazione. Alcuni tool infatti sono specifici per l'individuazione dei soli SNP, altri si focalizzano su mutazioni SNP ma possono essere estesi, con minime accortezze, ad individuare altri tipi di mutazioni, come indel (inserimenti e delezioni) o CNV (Copy number variation), altri ancora si concentrano su particolari mutazioni SNP \textit{de novo} (presenti nel genoma del figlio ma in nessuno dei due genitori).

In particolare, per le seguenti ragioni, sono stati selezionati i framework:
\begin{itemize} 
\item VarGeno (sezione \ref{vargeno}): \cite{sun-medvedev2018vargeno} si focalizza sul rilevamento di SNP ed è un diretto sviluppo e potenziamento di LAVA \cite{shajii2016lava}, un altro precedente tool dello stato dell'arte.
\item MALVA (sezione \ref{malva}): \cite{bernardini2019malva} è in grado di genotipizzare SNP e indel multi-allelici, anche in regioni del genoma ad alta densità, e di gestire efficacemente un numero grande di varianti.
\item FastGT (sezione \ref{fastgt}): \cite{pajuste2017fastgt} conta le frequenze di k-mer unici nei dati del genoma direttamente da dati ``grezzi" e utilizza queste informazioni per inferire i genotipi di varianti conosciute. 
\item COBASI (sezione \ref{cobasi}): \cite{gomez-romero2018cobasi} individua SNV utilizzando le informazioni provenienti dai cambiamenti nella coverage di sottostringhe uniche, ed è particolarmente indicato per individuare SNV \textit{de novo}. 
\end{itemize} 

\end{document}
