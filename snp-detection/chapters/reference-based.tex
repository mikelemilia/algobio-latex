% Overview:
%   Introduzione TeX subfile for the project.
%   Each subfile MUST start with the following line
%		\documentclass[../main.tex]{subfiles}

\documentclass[../main.tex]{subfiles}

\begin{document}

\section{Reference Based Frameworks}
\label{rbm}

In questa prima sezione si riporteranno e analizzeranno alcuni dei più attuali e significativi metodi che non utilizzano la costosa pratica dell'allineamento delle read ma che usano un genoma di riferimento per determinare mutazioni ed SNP ed effettuare la genotipizzazione. Alcuni tool infatti sono specifici per l'individuazione dei soli SNP, altri si focalizzano su mutazioni SNP ma possono essere estesi, con minime accortezze, ad individuare altri tipi di mutazioni, come indel (inserimenti e delezioni) o CNV (Copy number variation), altri ancora si concentrano su particolari mutazioni SNP \textit{de novo} \textcolor{BurntOrange}{(presenti nel genoma del figlio ma in nessuno dei due genitori).}

\noindent
\\
In particolare, per le seguenti ragioni, sono stati selezionati i framework:
\begin{itemize} 
\item[-] VarGeno (sezione \ref{vargeno}): \cite{sun-medvedev2018vargeno} si focalizza sul rilevamento di SNP ed è un diretto sviluppo e potenziamento di LAVA \cite{shajii2016lava}, un altro precedente tool dello stato dell'arte.
\item[-] MALVA (sezione \ref{malva}): \cite{bernardini2019malva} è in grado di genotipizzare SNP e indel multi-allelici, anche in regioni del genoma ad alta densità, e di gestire efficacemente un grande numero di varianti.
\item[-] FastGT (sezione \ref{fastgt}): \cite{pajuste2017fastgt} conta le frequenze di k-mer unici nei dati del genoma direttamente da dati ``grezzi" e utilizza queste informazioni per inferire i genotipi di varianti conosciute. 
\item[-] COBASI (sezione \ref{cobasi}): \cite{gomez-romero2018cobasi} individua SNV utilizzando le informazioni provenienti dai cambiamenti nella coverage di sottostringhe uniche, ed è particolarmente indicato per individuare SNV \textit{de novo}. 
\end{itemize} 

\end{document}
