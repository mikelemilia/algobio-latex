% Overview:
%   Introduzione TeX subfile for the project.
%   Each subfile MUST start with the following line
%		\documentclass[../main.tex]{subfiles}

\documentclass[../main.tex]{subfiles}

\begin{document}

\section{Introduzione}

\subsection{Il problema: la genotipizzazione e la rilevazione di mutazioni nel genoma}

L'obiettivo di questo progetto è analizzare lo stato dell'arte attuale per quanto riguarda il problema della ricerca di mutazioni (SNP) utilizzando metodi che non prevedono senza l'uso dell'allineamento (\textit{alignment-free}/\textit{mapping-free}). In particolare questo problema viene identificato con il termine di genotipizzazione, il processo per determinare quali varianti genetiche possiede un individuo.


Nell'ambito degli studi genetici, la scoperta e la caratterizzazione delle variazioni di sequenza nelle popolazioni umane è un'attività importante; in particolare è necessario analizzare in modo efficiente le variazioni di un individuo rispetto a un genoma di riferimento e ai dati disponibili sulle variazioni genomiche. 


Le variazioni analizzate possono essere di diverse tipologie: l'SNP\footnote{\textcolor{red}{\ Un polimorfismo a singolo nucleotide (spesso definito in inglese single-nucleotide polymorphism o SNP, pronunciato snip) è un polimorfismo, cioè una variazione, del materiale genico a carico di un unico nucleotide, tale per cui l'allele polimorfico risulta presente nella popolazione in una proporzione superiore all'1\%. Al di sotto di tale soglia si è soliti parlare di variante rara (in inglese single-nucleotide variant o SNV).} // controlla l'ultima frase, COBASI e FastGT cercano SNV sarebbe utile definirle}, Single Nucleotide Polymorphisms, è un polimorfismo, cioè una variazione rispetto ad un unico nucleotide, tale per cui l'allele polimorfico risulta presente nella popolazione in una proporzione superiore all'1\%; gli indel sono inserzioni o eliminazioni di una o più basi all'interno del genoma; altre variazioni sono CNV, Copy Number Variation, fenomeno in cui diverse sezioni del genoma si ripetono in numero variabile e i riarrangiamenti. I diversi metodi e framework progettati per effettuare la genotipizzazione solitamente si concentrano su determinati tipi di mutazioni.


Attualmente le tecnologie NGS, Next-generation sequencing, sono largamente utilizzate per lo studio delle variazioni nel genoma: queste tecnologie permettono di sequenziare grandi genomi in un tempo ristretto e hanno la capacità di sequenziare, in parallelo, milioni di frammenti di DNA. I dati di sequenziamento, le read, vengono poi utilizzate come input per i diversi metodi di genotipizzazione.



\subsection{La \textit{pipeline} standard e la tecnica \textit{mapping-free}}


La pipeline standard utilizzata per la genotipizzazione include l'allineamento delle read in input con una sequenza di riferimento, anch'essa in input, consentendo un certo numero di disallineamenti o indel. Le read mappate vengono quindi utilizzate per assegnare i genotipi; si utilizzano diversi strumenti di calcolo, come ad esempio tra i più usati, SAMtools o GATK (Genome Analysis Toolkit), che valutano l'allineamento delle read in ogni posizione lungo il genoma e assegnano un punteggio di confidenza per indicare la probabilità dell'esistenza di una variante. Questo si ottiene usando algoritmi di inferenza statistica, che sono necessari perché gli allineamenti imperfetti creano incertezza sulla posizione assegnata a gli errore di sequenziamento possono indurre false varianti. Si richiede inoltre un database di varianti conosciute, per poi arrivare ad assegnare ciascun genotipo attraverso le probabilità di ogni possibile genotipo calcolate sulla base dei dati osservati. Questi approcci, tuttavia, sono computazionalmente costosi e richiedono molto tempo, soprattutto per effettuare l'allineamento di sequenza e sono diventati quindi poco pratici per le applicazioni cliniche, dove il tempo è importante. 


\textcolor{red}{ho iniziato, devo continuare...}












\end{document}