% Overview:
%   Kevlar TeX subfile for the project.
%   Each subfile MUST start with the following line
%		\documentclass[../main.tex]{subfiles}

\documentclass[../main.tex]{subfiles}

\begin{document}

\section{Reference Free Frameworks}
\begin{flushleft}
\textbf{Migliorare e ampliare introduzione sui refence free frameworks.}
\linebreak
\end{flushleft}
I metodi reference-based sopracitati, non possono essere applicati quando non è presente un genoma di riferimento. 
In generale, i metodi reference-free specializzati nell’individuare SNP possono essere divisi in due macro categorie: 
\begin{itemize} 
\item i metodi della prima categoria eseguono inizialmente un assemblaggio de novo (de novo assembly) per costruire una sequenza di riferimento. Successivamente tramite un sotto-modulo (subroutine) reference based per mappare le letture (reads) di ogni individuo su questa nuova sequenza di riferimento. Ci si riferisce a questi metodi come ibridi (hybrid) siccome utilizzano sia assemblaggio de novo sia tecniche di mappatura (mapping techniques) per individuare SNP.
\item i metodi della seconda categoria si concentrano direttamente sull’assemblaggio degli SNP, senza cercare di assemblare preventivamente un riferimento completo (full reference)
\end{itemize}

\subsection{Kevlar}

\textbf{Scrivere introduzione}

\subsubsection{Data structure}
\subsubsection{Algorithms}
\subsubsection{Pipeline}

\end{document}