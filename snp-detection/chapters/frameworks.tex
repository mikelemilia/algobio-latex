% Overview:
%   Introduzione TeX subfile for the project.
%   Each subfile MUST start with the following line
%		\documentclass[../main.tex]{subfiles}

\documentclass[../main.tex]{subfiles}

\begin{document}

\section{Frameworks}
\label{frameworks}

In questa prima sezione, verranno proposti e analizzati alcuni dei più attuali e significativi framework \textit{alignment-free}/\textit{mapping-free} che si occupano di individuare e classificare mutazioni all'interno di read sequenziate da tecnologie NGS. Rispetto a ciascun framework, verranno analizzate le strutture dati e gli algoritmi su cui basano le chiamate delle varianti e i modelli statistici su sui operano la successiva genotipizzazione.

È utile far notare che alcuni dei metodi \textit{alignment-free}/\textit{mapping-free} che saranno presentati non si focalizzeranno esclusivamente sull'individuazione di SNP all'interno di un genoma ma, grazie ad alcuni accorgimenti, identificheranno un ampio spettro di mutazioni, come indel (inserimenti e delezioni) o CNV (Copy number variation), SNP \textit{de novo} e/o SNV. 

Con questo fine, differentemente dai classici metodi \textit{alignment-based} che utilizzano la costosa pratica dell'allineamento delle read per predire e/o individuare la presenza di mutazioni, il modello predominante su cui si baseranno i metodi che presenteremo sarà quello di operare su un genoma di riferimento e una lista preassegnata di SNP noti per effettuare la chiamata delle varianti all'interno delle read in input: ci riferiremo a questi metodi come \textit{reference-based}. La tecnica consiste nel confrontare le read in input con il genoma di riferimento, considerando la lista di mutazioni note, senza utilizzare costose tecniche di allineamento: solitamente vengono utilizzati \textit{k}-mer, ovvero sequenze di nucleotidi lunghe \textit{k}, ed effettuati dei conteggi rispetto a questi \textit{k}-mer per determinare, attraverso un modello probabilistico, il genotipo più probabile di ogni variante presente nelle read, che corrisponde a quello che presenta la maggiore probabilità.

È chiaro però che, anche quando esiste un genoma di riferimento, il comportamento dei metodi \textit{reference-based} è fortemente dipendente dalla qualità dell'assemblaggio e, in genere, l'utilizzo e l'analisi del riferimento richiede una grande quantità di tempo e risorse. Pertanto, non è possibile applicare un metodo \textit{reference-based} quando non esiste un genoma di riferimento e con questo s'intende che, dato il calo del costo richiesto per il sequenziamento, gli sforzi per sequenziare non sono più limitati alle principali specie di interesse (umano, primati, topi, etc.) e ad esempio, i biologi lavorano sempre più su dati per i quali non c'è alcun genoma di riferimento stretto. Per questi motivi, c'è una forte necessità di metodi \textit{reference-free} in grado di rilevare SNP, in particolare quelli isolati, senza fare affidamento su un genoma di riferimento. Questi ultimi, possono essere divisi in due macro categorie:
\begin{itemize} 
\item[-] metodi che eseguono inizialmente un assemblaggio \textit{de novo} per costruire una sequenza di riferimento e successivamente, tramite un sotto-modulo \textit{reference based}, mappano le read di ogni individuo su questa nuova sequenza di riferimento. Ci si riferisce a questi metodi come ibridi (\textit{hybrid}) poiché utilizzano sia assemblaggio \textit{de novo} sia tecniche di \textit{mapping} per individuare SNP.
\item[-] metodi che concentrano gli sforzi direttamente sull'assemblaggio degli SNP tramite l'utilizzo dei grafi di \textit{de Bruijn}, senza cercare di assemblare preventivamente un riferimento completo. Ci si riferisce a questi metodi come \textit{de novo}.
\end{itemize}

\noindent
\\Elenchiamo brevemente i framework che sono stati selezionati, e i rispettive motivi; essi verranno appunto analizzati in dettaglio. In particolare, i primi quattro metodi possono essere catalogati come metodi \textit{reference-based} mentre gli ultimi due \textcolor{red}{non usano un genoma di riferimento in senso stretto e si utilizzano tecniche \textit{reference-free} miste a tecniche \textit{reference-free}}.

\begin{itemize} 
\item[-] VarGeno (sezione \ref{vargeno}): \cite{sun-medvedev2018vargeno} si focalizza sul rilevamento di SNP ed è un diretto sviluppo e potenziamento di LAVA \cite{shajii2016lava}, un altro precedente tool dello stato dell'arte;
\item[-] MALVA (sezione \ref{malva}): \cite{bernardini2019malva} è in grado di genotipizzare SNP e indel multi-allelici, anche in regioni del genoma ad alta densità, e di gestire efficacemente un grande numero di varianti;
\item[-] FastGT (sezione \ref{fastgt}): \cite{pajuste2017fastgt} conta le frequenze di k-mer unici nei dati del genoma direttamente da dati ``grezzi" e utilizza queste informazioni per inferire i genotipi di varianti conosciute;
\item[-] COBASI (sezione \ref{cobasi}): \cite{gomez-romero2018cobasi} individua SNV utilizzando le informazioni provenienti dai cambiamenti nella coverage di sottostringhe uniche, ed è particolarmente indicato per individuare SNV \textit{de novo};
\item[-] Kevlar (sezione \ref{kevlar}): \cite{standage2019kevlar} si basa sulla ricerca di mutazioni \textit{de novo} tramite la comparazione diretta di sequenze appartenenti a individui imparentati;
\item[-] \textsc{DiscoSnp}\texttt{++} (sezione \ref{discosnp++}): \cite{peterlongo2017discosnp++} implementa un'efficiente strategia per analizzare e classificare le bolle all'interno di un grafo di \textit{de Bruijn}, riuscendo a predire SNV, isolati e prossimali, e indel;
\end{itemize} 

\end{document}

