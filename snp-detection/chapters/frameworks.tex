% Overview:
%   Introduzione TeX subfile for the project.
%   Each subfile MUST start with the following line
%		\documentclass[../main.tex]{subfiles}

\documentclass[../main.tex]{subfiles}

\begin{document}

\section{Frameworks}
\label{frameworks}

In questa prima sezione, verranno proposti e analizzati alcuni dei più attuali e significativi framework \textit{alignment-free}/\textit{mapping-free} che si occupano di individuare e classificare mutazioni all'interno di read sequenziate da tecnologie NGS. Per ciascun framework verranno analizzate le strutture dati e gli algoritmi su cui basano la chiamata delle varianti e i modelli statistici su sui operano la successiva genotipizzazione.

È utile far notare che alcuni dei metodi \textit{alignment-free}/\textit{mapping-free} che saranno presentati non si focalizzeranno esclusivamente sull'individuazione di SNP all'interno di un genoma ma, grazie ad alcuni accorgimenti, individeueranno un ampio spettro di mutazioni, come indel (inserimenti e delezioni) o CNV (Copy number variation), SNP \textit{de novo} e/o SNV. 


Per fare ciò, differentemente dai classici metodi \textit{alignment-based} che utilizzano la costosa pratica dell'allineamento delle read per predire e/o individuare la presenza di mutazioni, il modello predominante su cui si baseranno i metodi che presenteremo sarà quello di operare su un genoma di riferimento e una lista preassegnata di SNP noti per effettuare la chiamata delle varianti all'interno delle read e ci riferiremo a questi metodi come \textit{reference-based}. 

È chiaro però che non è possibile applicare un metodo \textit{reference-based} quando non esiste un genoma di riferimento. In effetti, anche quando esiste un genoma di riferimento, il comportamento di questi metodi è fortemente dipendente dalla qualità dell'assemblaggio. Al giorno d'oggi, infatti, con il calo del costo richiesto per il sequenziamento, gli sforzi per sequenziare non sono più limitati alle principali specie di interesse (umano, primati, topi, etc.) e i biologi lavorano sempre più su dati per i quali non c'è alcun genoma di riferimento stretto. Sfortunatamente, mentre il sequenziamento di NGS sta diventando una routine, assemblare i genomi rimane un compito molto complicato, per il quale nessun singolo software funziona costantemente bene, producendo così sequenze di riferimento di scarsa qualità. Per questi motivi, c'è una forte necessità di metodi \textit{reference-free} in grado di rilevare SNP, in particolare per quelli isolati, senza fare affidamento su un genoma di riferimento.

\noindent
\\
Dopo questa breve introduzione, per i seguenti motivi, sono stati selezionati i framework:
\begin{itemize} 
\item[-] VarGeno (sezione \ref{vargeno}): \cite{sun-medvedev2018vargeno} si focalizza sul rilevamento di SNP ed è un diretto sviluppo e potenziamento di LAVA \cite{shajii2016lava}, un altro precedente tool dello stato dell'arte.
\item[-] MALVA (sezione \ref{malva}): \cite{bernardini2019malva} è in grado di genotipizzare SNP e indel multi-allelici, anche in regioni del genoma ad alta densità, e di gestire efficacemente un grande numero di varianti.
\item[-] FastGT (sezione \ref{fastgt}): \cite{pajuste2017fastgt} conta le frequenze di k-mer unici nei dati del genoma direttamente da dati ``grezzi" e utilizza queste informazioni per inferire i genotipi di varianti conosciute. 
\item[-] COBASI (sezione \ref{cobasi}): \cite{gomez-romero2018cobasi} individua SNV utilizzando le informazioni provenienti dai cambiamenti nella coverage di sottostringhe uniche, ed è particolarmente indicato per individuare SNV \textit{de novo}.
\item[-] Kevlar (sezione \ref{kevlar}): \cite{standage2019kevlar} si basa sulla ricerca di mutazioni \textit{de novo} tramite la comparazione diretta di sequenze appartenenti a individui imparentati.
\item[-] \textsc{DiscoSnp}\texttt{++} (sezione \ref{discosnp++}): \cite{peterlongo2017discosnp++} implementa un'efficiente strategia per analizzare e classificare le bolle all'interno di un grafo di \textit{de Bruijn}, riuscendo a predire SNV, isolati e prossimali, e indel.
\end{itemize} 

\end{document}